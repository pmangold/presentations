\documentclass[aspectratio=169,12pt]{beamer}

\usepackage{hyperref}
\usepackage[giveninits=true,doi=false,isbn=false,url=false,eprint=false]{biblatex}



\renewbibmacro*{booktitle}{%
  \textbf{\printfield[booktitle]{booktitle}}%
}


\usepackage{amsmath}
\usepackage{bm}

\usepackage{tcolorbox}

\addbibresource{references.bib}
\addbibresource{publications_oral.bib}

\usepackage{varwidth}
\usepackage{tikz}
\usetikzlibrary{tikzmark}



\tikzset{
    blueblock/.style = {draw=amethyst,thick,fill=amethyst!15,rounded corners=0pt,inner sep=7pt}
}
\tikzset{
    pinkblock/.style = {draw=amaranth,thick,fill=amaranth!15,rounded corners=0pt,inner sep=7pt}
}

\tikzset{
    widepinkblock/.style = {draw=amaranth,thick,fill=amaranth!15,rounded corners=0pt,inner sep=20pt}
}

\usepackage{multirow}
\usepackage{booktabs}
\usepackage{algorithm}
\usepackage{algpseudocode}


\usepackage{amssymb}% http://ctan.org/pkg/amssymb
\usepackage{pifont}% http://ctan.org/pkg/pifont
\newcommand{\cmark}{\ding{51}}%
\newcommand{\xmark}{\ding{55}}%

\definecolor{darkspringgreen}{rgb}{0.09, 0.45, 0.27}
\definecolor{amaranth}{rgb}{0.9, 0.17, 0.31}
\definecolor{amethyst}{rgb}{0.6, 0.4, 0.95}

\usepackage{enumitem}

% Labels for items in (nested) itemize (uses bullets/characters)
\newcommand\labelitemi{\textcolor{amaranth}{\textbullet}}% bullet
\newcommand\labelitemii{\textcolor{amaranth}{\normalsize\normalfont\bfseries \textendash}}% --
\newcommand\labelitemiii{\textcolor{amaranth}{\textasteriskcentered}}% *
\newcommand\labelitemiv{\textcolor{amaranth}{\textperiodcentered}}% .
\setlist[enumerate,1]{label={\textcolor{amaranth}{\arabic*.}}}


% footnote size
\makeatletter
\newcommand\notsotiny{\@setfontsize\notsotiny\@vipt\@viipt}
\makeatother

\setbeamerfont{footnote}{size=\notsotiny}


\makeatletter
\newcommand{\Pause}[1][]{\unless\ifmeasuring@\relax
\pause[#1]%
\fi}
\makeatother
%%%%%% Template things

% space between paragraphs
\parskip=1em

% title font
\setbeamerfont{title}{size=\LARGE}%, series=\bfseries}
\setbeamerfont{frametitle}{size=\LARGE}%, series=\bfseries}
\setbeamerfont{institute}{size=\normalsize}%, series=\bfseries}

% spacing between frame title and content
\addtobeamertemplate{frametitle}{\vspace*{0.2cm}}{\vspace*{0.5cm}\setcounter{footnote}{0}}

% color
\definecolor{beamer@blendedblue}{rgb}{0.8, 0, 0.34}%{0.44, 0.16, 0.39}

% no navigation
\beamertemplatenavigationsymbolsempty

% slide numbers
\setbeamertemplate{footline}
{
  \hbox{\begin{beamercolorbox}[wd=1\paperwidth,ht=5.25ex,dp=4ex,right]{framenumber}%
      \large \insertframenumber{}~~
    \end{beamercolorbox}}%
  \vskip0pt%
}

% slide numbers
\def\setupappendix{
\setcounter{framenumber}{0}
\setbeamertemplate{footline}
{
  \hbox{\begin{beamercolorbox}[wd=1\paperwidth,ht=5.25ex,dp=4ex,right]{framenumber}%
      \large A-\insertframenumber{}~~
    \end{beamercolorbox}}%
  \vskip0pt%
}
}


% centered titles
\makeatletter
\long\def\beamer@@frametitle[#1]#2{%
  \beamer@ifempty{#2}{}{%
    \gdef\insertframetitle{\centering{#2\ifnum\beamer@autobreakcount>0\relax{}\space\usebeamertemplate*{frametitle continuation}\fi}}%
  \gdef\beamer@frametitle{#2}%
  \gdef\beamer@shortframetitle{#1}%
}%
}
\makeatother

%%%%% Equations

\newcounter{mytn}
\makeatletter
\newcommand{\tmn}[3][]{\stepcounter{mytn}%
\tikzmarknode[Col\the\numexpr\value{mytn}-\mytn@start\relax/.try,inner xsep=2pt,%
minimum height=1.6em,inner sep=2mm,#1]{mytn-\number\value{mytn}}{#2}%
\expandafter\gdef\csname tmn@annot@\number\value{mytn}\endcsname{#3}}
\newenvironment{AnnotatedEquation}{\edef\mytn@start{\number\value{mytn}}%
\begin{equation*}}{\end{equation*}%
\edef\mytn@end{\number\value{mytn}}%
\ifnum\mytn@end>\mytn@start
\begin{itemize}
 \foreach \X in {\the\numexpr\mytn@start+1,...,\mytn@end}
 {\item \tikzmarknode{mytn-annot-\X}{\csname tmn@annot@\X\endcsname}%
   \begin{tikzpicture}[overlay,remember picture]
  \draw[-stealth] (mytn-annot-\X.east) to[out=0,in=-90] (mytn-\X.south);
 \end{tikzpicture}}
\end{itemize}
\fi}
\makeatother
\tikzset{ Col1/.style= {fill=blue!20,anchor=base,rounded corners=2pt},
Col2/.style= {Col1, fill=red!20},
Col3/.style= {Col1, fill=green!20},
Col4/.style= {Col1, fill=yellow!20},
}

% footnotes at end of frame rather than minipage

\makeatletter
\renewrobustcmd{\blx@mkbibfootnote}[2]{%
  \iftoggle{blx@footnote}
    {\blx@warning{Nested notes}%
     \addspace\mkbibparens{#2}}
    {\unspace
     \ifnum\blx@notetype=\tw@
       \expandafter\@firstoftwo
     \else
       \expandafter\@secondoftwo
     \fi
       {\csuse{blx@theendnote#1}{\protecting{\blxmkbibnote{end}{#2}}}}
       {\csuse{footnote}[frame]{\protecting{\blxmkbibnote{foot}{#2}}}}}}
\makeatother

%%%%% INFORMATION

\title{\LARGE Régression Logistique \\[0.5em]
\large Mini-Cours
\vspace{0em}}
\author{
  Paul Mangold \\[0.5em]
  Audition pour le poste de Maître de Conférences à l'Université Dauphine-PSL
}
\titlegraphic{
}
\institute{}
\date{16 mai 2025}


%%%%% DOCUMENT

\begin{document}

%% TITLE PAGE

\begin{frame}[plain]
  \vspace{3.5em}
  \titlepage
\end{frame}
\addtocounter{framenumber}{-1}



% -------------------------
\begin{frame}{Pourquoi la régression logistique ?}
\begin{itemize}
  \item Objectif : Prédire une variable binaire (0 ou 1), ex : malade ou non, succès ou échec.
  \item Contrairement à la régression linéaire, la régression logistique :
  \begin{itemize}
    \item Modélise une probabilité.
    \item Contraint la sortie à $[0,1]$.
  \end{itemize}
  \item Utilisée massivement en médecine, épidémiologie, sciences sociales.
\end{itemize}
\end{frame}

% -------------------------
\begin{frame}{Formulation du modèle}
On modélise la probabilité que $Y = 1$ par :
\[
P(Y = 1 \mid X) = \sigma(w^T X) = \frac{1}{1 + e^{-w^T X}}
\]
où :
\begin{itemize}
  \item $\sigma$ est la fonction sigmoïde,
  \item $w$ est le vecteur de paramètres,
  \item $X$ est le vecteur de variables explicatives.
\end{itemize}
\end{frame}

% -------------------------
\begin{frame}{Qu’est-ce qu’un \textit{odds ratio} ?}
\begin{itemize}
  \item Définition des \textbf{odds} : 
  \[
  \text{odds} = \frac{P(Y=1 \mid X)}{P(Y=0 \mid X)} = \frac{p}{1-p}
  \]
  \item On a alors :
  \[
  \log\left(\frac{p}{1-p}\right) = w^T X
  \]
  \item L’odds ratio (OR) entre deux groupes $X$ et $X'$ vaut :
  \[
  \text{OR} = \exp(w^T (X - X'))
  \]
\end{itemize}
\end{frame}

% -------------------------
\begin{frame}{Importance des odds ratios en recherche médicale}
\begin{itemize}
  \item Interprétation simple : OR > 1 $\Rightarrow$ le facteur augmente le risque.
  \item Exemple :
  \begin{itemize}
    \item OR = 3.2 pour un certain médicament $\Rightarrow$ 3.2 fois plus de chances de succès.
  \end{itemize}
  \item Donne un outil quantitatif pour identifier des facteurs de risque.
\end{itemize}
\end{frame}

% -------------------------
\begin{frame}{Extension au multi-classe : le softmax}
Pour $K$ classes :
\[
P(Y = k \mid X) = \frac{e^{w_k^T X}}{\sum_{j=1}^K e^{w_j^T X}}
\]
\begin{itemize}
  \item Chaque classe $k$ a un vecteur $w_k$.
  \item Cette fonction s’appelle \textbf{softmax}.
  \item Utilisé pour généraliser la régression logistique à la classification multi-classes.
\end{itemize}
\end{frame}

% -------------------------
\begin{frame}{La fonction de perte logistique}
Pour un seul échantillon $(x_i, y_i)$ :
\[
\ell(w; x_i, y_i) = -\left[ y_i \log \hat{p}_i + (1 - y_i) \log (1 - \hat{p}_i) \right]
\]
où $\hat{p}_i = \sigma(w^T x_i)$

\begin{itemize}
  \item Cette fonction pénalise fortement les erreurs de prédiction.
  \item Convexe et différentiable : propice à l’optimisation par descente de gradient.
\end{itemize}
\end{frame}

% -------------------------
\begin{frame}{Dérivation via maximum de vraisemblance}
On suppose que :
\[
P(y_i \mid x_i; w) = \sigma(w^T x_i)^{y_i} (1 - \sigma(w^T x_i))^{1 - y_i}
\]
La vraisemblance totale est :
\[
L(w) = \prod_{i=1}^n P(y_i \mid x_i; w)
\]
On maximise le log :
\[
\log L(w) = \sum_{i=1}^n \left[ y_i \log \hat{p}_i + (1 - y_i) \log (1 - \hat{p}_i) \right]
\]
Donc la \textbf{perte logistique} est l’opposée du log-vraisemblance.
\end{frame}

% -------------------------
\begin{frame}{Résumé}
\begin{itemize}
  \item La régression logistique modélise $P(Y = 1 \mid X)$ via une fonction sigmoïde.
  \item L’odds ratio est une mesure clé pour l’interprétation des effets des variables.
  \item Le softmax permet l’extension à plusieurs classes.
  \item La fonction de perte logistique provient du maximum de vraisemblance.
\end{itemize}
\end{frame}

% -------------------------
\begin{frame}{Questions ?}
\centering
\Huge Merci de votre attention !
\end{frame}

\end{document}
%%% Local Variables:
%%% mode: latex
%%% TeX-master: t
%%% End:
